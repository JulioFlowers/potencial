Las figuras \ut{\ref{fig:pot1} - \ref{fig:pot10}} muestran el comportamiento de la corriente en función de . En todas ellas se $V^{\nicefrac{3}{2}}$identifican claramente tres regiones clave: la zona regida por la ley de Child-Langmuir, la zona de transición y la zona de saturación. Además, para los diez barridos, se señala el punto ${V_{p}}^{\nicefrac{3}{2}}$, asociado a la primera energía de ionización. Esto se debe a que, por definición, el electronvoltio representa la energía necesaria para mover un electrón en un campo eléctrico con una diferencia de potencial de 1 V. En este caso, el potencial medido se relaciona con la energía requerida para acelerar un electrón y provocar colisiones elásticas con los electrones del gas en el bulbo, lo que resulta en la liberación de un electrón de la capa de valencia del gas.

La tabla \ut{\ref{tab:pots}} presenta los potenciales de ionización determinados mediante el criterio de la tercera derivada y derivación numérica, junto con sus intervalos de confianza. Se observa que los potenciales de ionización calculados no aparecen inmediatamente después de la región lineal, sino tras un primer cambio de curvatura. Esto ocurre porque algunos electrones generados por efectos termoiónicos poseen una energía cinética inicial suficiente para, al alcanzar un cierto potencial (en este caso, 9 V), superar la energía de ionización del gas. Este fenómeno provoca la ionización de unos pocos átomos y, como consecuencia, la pérdida de linealidad en las gráficas \ut{\ref{fig:pot1} - \ref{fig:pot10}}.


Se obtuvo un valor de energía de ionización promedio de $V_{p}\ =\ 	10.59\ eV,\ \sigma\ =\ 0.140 eV$, el cual se aproxima con un error porcentual $1.54\ \%$ al potencial de ionización de Mercurio ($10.43eV$) {\color{Mulberry} cita NIST}, cuyos vapor es utilizado en el diseño de tiratrones. Las figuras \ut{\ref{fig:derspot1} - \ref{fig:derspot10}}, y \ut{\ref{fig:der3pot1} - \ref{fig:der3pot10}}  muestran la primera segunda y tercer derivada para cada una de la regiones denominadas como \textit{Ajuste rodilla LOESS} para las figuras, \ut{\ref{fig:pot1} - \ref{fig:pot10}}, en ellas se puede apreciar que a pesar de que todos los puntos en esta región fueron utilizados para realizar el ajuste por método de LOESS, debido a la presencia de ruido en las mediciones esta curva presenta varios puntos de inflexiones consecuencia de buscar una  la 

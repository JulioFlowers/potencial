
\begin{figure}[h!]
	\centering
	\includegraphics[width=\columnwidth]{img/circuitopot.png}
	\caption{Circuito construido con un tiratrón 2D21, donde se aprecian la fuente de Voltaje DC (V1), con un voltaje variable de 1.2 -12 V, así mismo se puede apreciar los puntos donde se registra corriente y voltaje, a la par de los puntos F los cuales fueron conectados a una fuente de voltaje alterna de 6 V.}
	\label{fig:circuit}
\end{figure}

La figura \ut{\ref{fig:circuit}} muestra la configuración del tiratrón, así como las secciones del circuito donde se registra el voltaje asociado a la aceleración de los electrones desprendidos por fenómeno termoiónico, y la corriente a través del ánodo.

Se realizó un primer barrido de la corriente en función del voltaje aplicado, con la finalidad de detectar donde inicia la región de transición entre la zona de descriptible por la ley de Child - Langmuir, y la región de saturación. Posteriormente se realizaron 10 barridos de corriente con respecto al voltaje, en los cuales se realizo un barrido grueso (cambios de 0.5 V o 1 V) en la región dominada por la ley de Child - Langmuir; así mismo en la región de transición la resolución de los barridos fue modificada a 0.1 V. 

Las diez series de datos fueron linealizados aplicando ley de Child - Langmuir a todos los datos (i,e. el voltaje registrado fue elevado a la potencia de $\frac{3}{2}$ ), se seleccionaron los primeros n puntos los cuales mostraran un comportamiento lineal, y a la par que permitieran generar un ajuste lineal mediante mínimos cuadrados que maximicé $R^{2}$, a su vez una recta fue trazada entre dos puntos en la zona de saturación. Por otra parte una curva suavizada fue ajustada a los datos medidos mediante el Método de LOESS utilizando una equipartición de puntos discretos con distancias iguales a la mínima escala del multímetro utilizado (0.01 V) . La intersección entre la curva y la recta generada por mínimos cuadrados así como la intersección entre la curva y la recta de la zona de saturación son tomados como la zona de transición y un nuevo ajuste por LOESS, donde se contemplan todos los puntos como venta (mejorando asi la suavidad de la curva, sin modificar la precisión al ajustar los datos, y reduciendo el tiempo computacional) es realizado. A partir del uso de diferencias finitas de cuarto orden se calculan la segunda y tercera derivada, y se busca un punto $V_{p}$ dentro del dominio de $f^{\prime \prime}$, de tal forma que $f^{\prime \prime}(V_{p})$ se acerque al cero con una tolerancia de $1\times10^{-5}$, y  $f^{(3)}(V_{p})$ sea diferente de cero, de tal forma que por el criterio de la tercer derivada se encuentre el punto de inflexión que precede a la zona de saturación el cual es potencial de ionización. 

Finalmente se establece un intervalo de confiabilidad calculando la distancia promedio de $V_{p}$  al punto del dominio encontrado por la intersección de la recta generada por mínimos cuadrados y el primer ajuste por LOESS, así como el punto generado por la intersección de la recta del área de saturación y la recta generada por mínimos cuadrados. 

Dentro de los objetivos propuestos, fue posible encontrar el gas dentro del tiratrón siendo este el vapor de mercurio con un error de 1.54 \% con respecto a la literatura. Aunque los resultados obtenidos fueron consistentes con la literatura, el proceso de derivación numérica, especialmente en la tercera derivada, mostró la presencia de ruido, y errores asociados a las limitaciones del equipo de medición. Este ruido introdujo puntos de inflexión adicionales en las curvas, lo que dificultó la precisión en la determinación del punto de inflexión exacto asociado al voltaje de ionización. A pesar de esto, se estableció un intervalo de confiabilidad para el potencial de ionización, lo que sugiere que si se repitiera el experimento con otro conjunto de datos, el valor estaría dentro de este rango.

Para futuros acercamientos al problema, se recomendaría reducir el ruido asociado a las mediciones a la par de una mejora en la resolución de los equipos de adquisición de datos. Para este caso una fuente cuyos cambios de voltaje sean de mayor precisión y repetibles, como un potenciómetro digital con un amplificador operacional de baja deriva en configuración de seguidor de voltaje, a la par de un un multimétro de banco con mayor resolución y precisión. Esto permitiría obtener mediciones más precisas y constantes, reduciendo el ruido y haciendo posible un análisis más exacto de las derivadas numéricas, lo que optimizaría la identificación del punto de ionización. Con este enfoque, se podrían obtener resultados más confiables y reproducibles, reduciendo el intervalo de confiabilidad.